%%%%%%%%%%%%%%%%%%%%%%%%%%%%%%%%%%%%%%%%%%%%%%%%%%%%%%%%%%%%%%%%%%%%%%%%%%%%%
\section{GPGPU Introduction}

-Cos'�
-perch�
---Tabella prestazioni
---Crescita maggiore della legge di Moore
-esempi applicativi
-problemi: difficolt�

%-----------------------------------------------------------------------------
\subsection{Basic Principles}
% Please avoid separations in titles
% and separate text manually


\begin{itemize}
	\item Array \= Texture
	\item Kernels
	\item Feedback
	\item ....
\end{itemize}


%This is a citation \cite{Norman09Learn} and here is another citation
%\cite{Peyton93Howto}.  Lorem ipsum dolor sit amet, consectetur adipiscing elit.


%And this is the reference to a single column figure (see {\bf Figure
%\ref{fig:myfigure1}}).  Lorem ipsum dolor sit amet, consectetur adipiscing elit.

\begin{figurehere}
 \centering
 \includegraphics[width=8cm, height=4cm]{./eps/placeholder.eps}
 \caption{Some single-column figure caption.}
 \label{fig:myfigure1}
\end{figurehere}


%-----------------------------------------------------------------------------
\subsection{GPGPU Programming Languages}


\begin{figure*}[t]
  \centering
 \includegraphics[width=16cm, height=4cm]{./eps/placeholder.eps}
 \caption{Some wide-figure caption.}
 \label{fig:myfigure2}
\end{figure*}

And this is the reference to a single column figure (see {\bf Figure
\ref{fig:myfigure2}}). Lorem ipsum dolor sit amet, consectetur adipiscing elit.

%%%%%%%%%%%%%%%%%%%%%%%%%%%%%%%%%%%%%%%%%%%%%%%%%%%%%%%%%%%%%%%%%%%%%%%%%%%%%