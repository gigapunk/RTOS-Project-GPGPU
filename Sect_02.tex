%%%%%%%%%%%%%%%%%%%%%%%%%%%%%%%%%%%%%%%%%%%%%%%%%%%%%%%%%%%%%%%%%%%%%%%%%%%%%
\section{OpenCL and Device Fission}

OpenCL consists of an API for coordinating \textit{parallel computation across heterogeneous processors} (CPU, GPU and other processors) and it is supported by a wide range of systems and platforms, making it the perfect choice for parallel computation not only on traditional desktop CPU-GPU configuration, but also on embedded systems.


%-----------------------------------------------------------------------------
\subsection{OpenCL Architecture}

OpenCL is a framework which is composed of 4 main components:

\begin{enumerate}
	\item a \textbf{language} [APPROFONDIRE]
	\item an \textbf{API} [APPROF]
	\item a series of \textbf{libraries} [APPROF]
	\item a \textbf{runtime system} [APPROF]
\end{enumerate}

To better describe the architecture of OpenCL, we can divide it into four models:

\subsubsection{The Platform Model}
We can define the main components of an OpenCL application:

\begin{itemize}
	\item the \textbf{Host} can be viewed as the "`outer control logic"' of the application. It is usually executed on the CPU and its function is to configure the application accordingly to the architecture of the hosting machine, and to submit commands to the computing units.
	\item one or more \textbf{OpenCL Devices} connected to the host. These devices can be physical (e.g. the graphic adapter installed on the system) or virtual (e.g. remote GPUs in a cluster configuration (www.mosix.org))
\end{itemize}



%-----------------------------------------------------------------------------
\subsection{The second subsection of the second \\ Section}

Lorem ipsum dolor sit amet, consectetur adipiscing elit. Mauris eget mauris.
Nulla facilisi. Ut condimentum tempor eros? Integer metus mauris, consectetur
sit amet, tempor a, facilisis eu, nisl. Vestibulum at turpis. Ut vitae tortor
pretium nisl vestibulum blandit. Nulla nibh urna, semper et, elementum at,
mattis ut, nisi! Cum sociis natoque penatibus et magnis dis parturient montes,
nascetur ridiculus mus. Morbi vel ligula eget lacus convallis venenatis. Aliquam
lacinia tincidunt felis. Ut dui.
