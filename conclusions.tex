\section{Conclusions}

Parallel computing is a very effective mean for achieving higher performances in modern computing and, thanks to their architecture, modern graphic adapters are the best choice for the implementation of parallel applications. Since GPU programming is quite different from the standard programming approach, dedicated APIs and framework such as CUDA and OpenCL are required. In the first chapter we presented an introduction to the general concepts behind the GPGPU program model, and then we focused on the OpenCL API. We introduced concepts like host, kernels, memory objects and we described how they should be implemented using some examples. In the second part of the second chapter, we talked about an important feature introduced by the last implementation of the OpenCL API: Device Fission.\\ Device Fissioning gives to the programmer total control over its application and allow him to better manage resources and power consumption, and this would be extremely useful in those scenarios where resources are very limited and power is crucial, such as in an embedded environment. To conclude the Device Fission introduction, we provided some useful example scenarios in which it can be applied.\\
In the third chapter we finally analized some real world applications as well as some research studies that involved OpenCL and device fissioning, and we saw that such framework can be very useful in a wide area of applications, especially in commercial situations where time-to-market is crucial or where power consumption and cooling costs are a limitation for expansion. 